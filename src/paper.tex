% This is the main file for the content.
% Here go only includes.

\chapter{Why OSv}

    OSv is one out of many unikernel operating systems:

        \begin{itemize}
            \item Osv
            \item Mirage OS
            \item Drawbridge
            \item HalVM
            \item ErlangOnXen
            \item GUK
            \item NetBSD
            \item ClickOS
        \end{itemize}\cite[S. 13]{unikernels}

    But Anil Madhavapeddy and David J. Scott list it in their article
    ``Unikernels: Rise of the Virtual Library Operating System'' on unikernels
    as only unikernel operating system which can be run on the Kernel Virtual
    Machine (KVM).

    As the \HFU{} only uses KVM as
    hypervisor, the research on the whole topic of unikernels had to start with
    an unikernel operating system which can be run on KVM.

\chapter{The setup}\label{chap:setup}

    The testing machine specifications where not very good and not compareable
    to the cloud service machines the \HFU{} runs, but it was good enough:

        \begin{itemize}
            \item 4 GB RAM
            \item AMD Athlon Dual Core Processor 5400B
        \end{itemize}

    The test system was a Debian (wheezy, stable) installation at first. But it
    turned out that this was not the best idea, as much software had to be
    installed from the backport repositories of Debian, such as gcc-4.8.

    The second approch worked really good, using a Fedora (20) installation with
    the following specifications:

        \begin{itemize}
            \item Linux Kernel 3.13.3-201.fc20.x86\_64
            \item yum 3.4.3
        \end{itemize}

    \section{Installed packages}

        For compiling OSv from source, as well as compiling modules for OSv from
        source, we had to install a bunch of software.

        The following list of software is listed in the OSv documentation as
        required for building OSv:

            \begin{tabular}{ll}
                Package             & Version   \\
                \hline

                ant                 & 1.9.2     \\
                autoconf            & 2.69      \\
                automake            & 1.13.4    \\
                boost-static        & 1.54.0-9  \\
                gcc-c++             & 4.8.2     \\
                genromfs            & 0.5.2-10  \\
                libvirt             & 1.1.3.3-5 \\
                libtool             & 2.4.2-23  \\
                flex                & 2.5.37-4  \\
                bison               & 2.7-3     \\
                qemu-system-x86     & 2:1.6.1-3 \\
                qemu-img            & 2:1.6.1-3 \\
                maven               & 3.1.1     \\
                maven-shade-plugin  & 2.1-1     \\
            \end{tabular}

        But we installed some more packages, as we needed them to build OSv or
        the modules:

            \begin{tabular}{ll}
                Package             & Version   \\
                \hline

                boost               & 1.54.0-9  \\
                cmake               & 2.8.12.2  \\
                dhclient            & 12:4.2.8-1 \\
                git                 & 1.8.5.3   \\
                make                & 3.82      \\
                openssh-server      & 6.4       \\
                qemu-kvm            & 2:1.6.1-3 \\
            \end{tabular}

    \section{The repositories}

        We cloned the repositories of OSv and the OSv applications directly from
        the git repository they are hosted with:

\begin{lstlisting}
git clone https://github.com/cloudius-systems/osv
git clone https://github.com/cloudius-systems/osv-apps
\end{lstlisting}

        We then had to go to the latest version of OSv which was marked as
        stable, which was ``v0.05'' at the time we did this. The application
        repository had no version tags at this time.

        We had to update the submodules of the repository of OSv. This can be
        done by

\begin{lstlisting}
git submodule update --init
\end{lstlisting}

        in the repository.

    \section{Network setup}

        Before OSv can be run, the network has to be configured to allow the
        KVM/qemu to create virtual interfaces, which are required for external
        communication from within OSv.

        Our setup used ``virbr0'' (as discribed in the OSv documentation) as
        bridge and let ``em1'' (which equals ``eth1'' on Debian) unconfigured.
        The used configuration scripts are located in

\begin{lstlisting}
/etc/sysconfig/network-scripts
\end{lstlisting}

        in Fedora. The configuration scripts can be found in the appendix
        \ref{netconf:em1} (for ``em1'') and \ref{netconf:virbr0} (for
        ``virbr0``).

        After these configration files are put in the right directory, the
        ``NetworkManager'' has to be stopped via

\begin{lstlisting}
sudo /etc/init.d/NetworkManager stop
\end{lstlisting}

        and the network has to be restarted

\begin{lstlisting}
sudo /etc/init.d/network restart
\end{lstlisting}

        After that, the ``ifconfig'' command should show the ``em1'' interface
        as unconfigured and without an IP adress, the virbr0 interface instead
        has an IP. An example can be found in appendix \ref{example:ifconfig}.


    \section{SELinux}

        We also turned off SELinux. We don't know if it works if SELinux is
        enabled.

\chapter{Building OSv}

    Once the software packages are installed, the repository of OSv is
    cloned (the osv-apps repository is not required for a basic build) and
    the submodules of the repository are updated, the build of an image of
    OSv is really simple. By executing

\begin{lstlisting}
make
\end{lstlisting}

    an image gets build. This can take several minutes. On our testing
    system, the build process took about 20 minutes.

\chapter{Running OSv}

    Now, the image which just got build can be run. The OSv developers ship a
    Python script with the source repository, which can start OSv with qemu and
    KVM whichout writing down several commandline arguments every time.

    By executing

\begin{lstlisting}
python scripts/run.py
\end{lstlisting}

    OSv gets started. If the machine, OSv should run on, has just little RAM or
    not more than two processor cores, we recommend the following call:

\begin{lstlisting}
python scripts/run.py -c 1 -m 1G
\end{lstlisting}

    Where ``1G'' stands or one gigabyte RAM for the virtual machine and the
    ``1'' after  ``-c'' stands for one processor core.

    The script can print a help summary with the option ``--help'' or ``-h''.

\chapter{The setup script}

    With appendix \ref{install:osv}, we provide a script which can be run on a
    Fedora installation for automatically configurate the host and compile OSv
    from source. The script builds the latest version of OSv.

    It also needs some prerequisites:

        \begin{itemize}
            \item Packages
            \begin{itemize}
                \item bash
                \item sed
                \item systemctl
                \item tee
                \item yum
            \end{itemize}

            \item A user ``osvuser'' on the system. Can be configured inside the
            script
        \end{itemize}

    The packages should be provided by the Fedora installation. You may also
    need the ``wget'' or ``curl'' program, to retreive the script itself from an
    external host, if you don't want to use a flashdrive.

    We highly recommend a freshly deployed Fedora installation for this. The
    installation can take a lot of time, whereas we recommend a good machine, at
    least with 4GB RAM and a Dualcore in it.

    The script pulls in some dependencies in manner of packages it installes.
    This is wanted.

\chapter{Production usage}

    When it comes to production usage or stability, we wouldn't recommend using
    OSv, at least for the \HFU.

    There are several points why we wouldn't recommend it (yet):

    \section{Missing shell commands}

        The ``CRaSH'' shell, which is a Java implementation of a shell for OSv
        has only few commands available. While we did our research on the topic,
        it got improved a lot, but essential commands or functionalities as

            \begin{itemize}
                \item The ``mkdir'' command
                \item The ``rmdir'' command
                \item Pipes
                \item Redirections
            \end{itemize}

        and others are not available yet.

    \section{Shutdown when program exits}

        We tried porting a Java implementation of git to OSv. It worked quiet
        good, but there was one big issue: git returns with status codes which
        are not equal to zero, if some unexpected things happened.

        An exit code from an application which exits with non-zero return values
        shuts the VM down.

        Returning non-zero exit codes is a design decision by the git (and jgit)
        developers. We would be
        able to patch the jgit implementation to return zero all the time, but
        we are not able to patch it for every case it returns. In addition,
        external programs use the return codes of git for identification what
        happened. There would be no possibility to use (for example) gitolite, a
        security layer for git servers, with a patched jgit implementation.

    \section{Starting of applications}

        OSv is not able to start normal applications. They must be compiled as
        shared object. This is bad, because porting (for example) a real git
        implementation (not a Java implementation) to OSv would require us to
        port perl. Nobody wants to port perl to something!

    \section{External storage}

        We were not able to figure out how to plug external storage into OSv.

    \section{Documentation}

        The OSv documentation lags of basic information. It is all in all not
        very good or structured and partially wrong.

\chapter{Conclusion}

    OSv is activly maintained and developed. But at the current state of
    development and with the currently ``stable'' version, we wouldn't recommend
    OSv at all.

    Anyways, we wrote a script which installes a complete environment for OSv.
    It can be found in the appendix. It must be executed on a new and clean
    Fedora installation, which has the specifications we mentioned in the setup
    chapter \ref{chap:setup}.
