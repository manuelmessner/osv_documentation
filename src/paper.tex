% This is the main file for the content.
% Here go only includes.

\chapter{Why OSv}

    OSv is one out of many unikernel operating systems:

        \begin{itemize}
            \item Osv
            \item Mirage OS
            \item Drawbridge
            \item HalVM
            \item ErlangOnXen
            \item GUK
            \item NetBSD
            \item ClickOS
        \end{itemize}\cite[S. 13]{unikernels}

    But Anil Madhavapeddy and David J. Scott list it in their article
    ``Unikernels: Rise of the Virtual Library Operating System'' on unikernels
    as only unikernel operating system which can be run on the Kernel Virtual
    Machine (KVM).

    As the \HFU{} only uses KVM as
    hypervisor, the research on the whole topic of unikernels had to start with
    an unikernel operating system which can be run on KVM.

\chapter{The setup}

    The testing machine specifications where not very good and not compareable
    to the cloud service machines the \HFU{} runs, but it was good enough:

        \begin{itemize}
            \item 4 GB RAM
            \item AMD Athlon Dual Core Processor 5400B
        \end{itemize}

    The test system was a Debian (wheezy, stable) installation at first. But it
    turned out that this was not the best idea, as much software had to be
    installed from the backport repositories of Debian, such as gcc-4.8.

    The second approch worked really good, using a Fedora installation with the
    following specifications:

        \begin{itemize}
            \item Linux Kernel 3.13.3-201.fc20.x86\_64
            \item yum 3.4.3
        \end{itemize}


    \section{Installed packages}

        For compiling OSv from source, as well as compiling modules for OSv from
        source, we had to install a bunch of software.

        The following list of software is listed in the OSv documentation as
        required for building OSv:

            \begin{itemize}
                \item ant
                \item autoconf
                \item automake
                \item boost-static
                \item gcc-c++
                \item genromfs
                \item libvirt
                \item libtool
                \item flex
                \item bison
                \item qemu-system-x86
                \item qemu-img
                \item maven
                \item maven-shade-plugin
            \end{itemize}

        But we installed some more packages, as we needed them to build OSv or
        the modules:

            %\begin{itemize}
                % TODO
            %\end{itemize}

    \section{The repositories}

        We cloned the repositories of OSv and the OSv applications directly from
        the git repository they are hosted with:

\begin{lstlisting}
git clone https://github.com/cloudius-systems/osv
git clone https://github.com/cloudius-systems/osv-apps
\end{lstlisting}

        We then had to go to the latest version of OSv which was marked as
        stable, which was ``v0.05'' at the time we did this. The application
        repository had no version tags at this time.

        We had to update the submodules of the repository of OSv. This can be
        done by

\begin{lstlisting}
git submodule update --init
\end{lstlisting}

        in the repository.

    \section{Network setup}

        Before OSv can be run, the network has to be configured to allow the
        KVM/qemu to create virtual interfaces, which are required for external
        communication from within OSv.

        Our setup used ``virbr0'' (as discribed in the OSv documentation) as
        bridge and let ``em1'' (which equals ``eth1'' on Debian) unconfigured.
        The used configuration scripts are located in

\begin{lstlisting}
/etc/sysconfig/network-scripts
\end{lstlisting}

        in Fedora. The configuration scripts can be found in the appendix
        \ref{netconf:em1} (for ``em1'') and \ref{netconf:virbr0} (for
        ``virbr0``).

        After these configration files are put in the right directory, the
        ``NetworkManager'' has to be stopped via

\begin{lstlisting}
sudo /etc/init.d/NetworkManager stop
\end{lstlisting}

        and the network has to be restarted

\begin{lstlisting}
sudo /etc/init.d/network restart
\end{lstlisting}

        After that, the ``ifconfig'' command should show the ``em1'' interface
        as unconfigured and without an IP adress, the virbr0 interface instead
        has an IP. An example can be found in appendix \ref{example:ifconfig}.


\chapter{Building OSv}

    Once the software packages are installed, the repository of OSv is
    cloned (the osv-apps repository is not required for a basic build) and
    the submodules of the repository are updated, the build of an image of
    OSv is really simple. By executing

\begin{lstlisting}
make
\end{lstlisting}

    an image gets build. This can take several minutes. On our testing
    system, the build process took about 20 minutes.
