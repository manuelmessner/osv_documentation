\chapter{Introduction}

    Cloud computing and the ability to host several services and
    applications within an operating system on just one physical machine has
    gained a lot of attention over the past years. The ability to mutliplex
    virtual computers onto one (or more) physical machines is a common approach
    when it comes to cloud services. Services and providers for cloud hosting
    and virtual machines are long-established. Amazon or Rackspace host millions
    of virtual machines every day \cite[S. 1]{unikernels}.

    The \HFU{} runs a cloud for research as well as for the students. Students
    and employees are able to start a VM via a Webinterface, where they can
    install software they need for their work, test software or try new
    operating systems. They have two different Debian images, one with the LAMP
    software package preinstalled, the other one is a minimal Debian
    installation. They also
    have a option for starting an OwnCloud server and they provide a Windows
    virtual machine.

    As it is very resource-intensive to host several virtual machines which do
    basically the same - running a Debian with all its packages, services and
    tools - they wanted to try something new.

    The idea behind an unikernel is, that there is only the kernel of the
    operating system, the basic services which are required to run a service and
    the service itself. Computing time intensive protocols and layers, support
    for physical disk drives, SSDs (Solid state drive) or runtime environments
    for languages like Java or .NET are not included in the unikernel, as there
    is no need for them anymore \cite[S. 1]{unikernels}.

    An unikernel operating system includes the userspace code and the kernel
    code in flexible modules which are only runable within the VM itself.

    The \HFU{} wanted the authors of this paper do to research on this topic and
    to find out whether an unikernel operating system could be an alternative to
    some approches the virtual machines at the \HFU{} are used for. The \HFU{}
    wanted OSv to be tested, as they're using KVM for their cloud infrastructure
    and OSv is the one unikernel operating system which can run on KVM.
